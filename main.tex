\documentclass{ltjsarticle}
\usepackage{tikz}
\usepackage{./mystyle}


\begin{document}
\section{物理の考え方を生かす、ということ}
僕は高校生から大学院博士課程まで物理を学び続けたわけだが、正直修士課程の途中から日々の研究が苦痛になっていた。"誰も作ったことがない顕微鏡を作る"という僕の研究テーマだけを聞くと、とても面白そうに聞こえる。ただ、本当にそう"聞こえる"だけで、僕にとって絡まった糸をほどくような、終わりの見えない苦行の連続だった。高校生だった頃の物理への熱量は完全に冷め(冷めたどころか、エネルギーはほぼゼロだったかもしれない)、とうとう今年ソフトエンジニアとしてのキャリアをスタートした。ここで困ったのは、僕は周りと違って今まで積み重ねてきた知識や技術をそのまま応用することができないことだ。細胞に光を当てると蛍光を出す、光を当てるだけで水の温度が測れる、干渉系を用いた分光器を作れる、こんな話とソフトウエア分野はかすりもしていない。当然、内定を承諾した段階でこんなことは分かっていた。そして、自分は物理とはきっぱり縁を切る、と。そう決断していた。...ただ、、、勿体無い、という気持ちは心のどこかにあったのかもしれない。そしてあろうことか、CSの研修講義でこの未練の心に火がついてしまった。

\begin{tcolorbox}
    $n$枚のカードに、整数がそれぞれ書かれています。2枚目には $a_i$が書かれています($1\leq i \leq n$)。この中からちょうど3枚のカードを重複を許して選ぶ方法は全部でれ$n^3$通りありますが、この中でカードに書かれた整数の和が$k$になるような場合の数を求めてください。
\end{tcolorbox}
CS講座で取り組んでいた問題は、今まで大学で解いてきた問題とは全くの別物だった。AtCoderは若干齧ったことがある程度で"ほぼ初めまして"状態だったけど、この新しい脳の使い方は、知れば知るほど好奇心を刺激される。特に上記の問題はFourier変換で解ける、という事実には夢中になってしまった。大学-大学院と、光をテーマに自然を掘り下げてきたこともあり、Fourier変換には時々お世話になっていた。僕の実験では色んな色の光を扱う必要があったので、むしろ波数空間で物を考える方が多かったかも知れない。なのでFourier変換が"情報を波数空間で扱う解析手法"であることは人並み程度には理解している。こんな感じに中途半端に知識があるものだから、波の香りが一切しないにもかかわらずこのアルゴリズムの問題の解法にFourier変換が出てくるのは本当に奇妙で仕方がない。この問題に出会った日以降、退勤後はFourier変換の学び直し時間になった。今この文章を書いている時点ではこの疑問に対する回答は出ていて、"インパルス応答による測定結果の鈍りと関連しているから"が結論だ。文章にしてしまえば"何だこれだけか"感は拭えないし、事実、予想より遥かに単純な話だった。ただ、ここに至るまでに大きな学びがあって、"物理の考え方をどう生かすか"に解を一つ与えてくれた。
\\
\\
前述の通り、カードのアルゴリズム問題に出会って以降Fourier変換の学び直しに多くの時間を割いていた。この過程でいくつか解説記事を訪れてきたわけだが、ここで物理の参考資料との大きな違いに気づいた。それは、CSの記事は、物理に比べて数学の比重が大きいということだ。一応断っておくと、これはFourier変換の記事に限った話かもしれない。ただ少なくとも、Fourier変換の文脈に限れば間違いないと思う。たとえば"Fourier変換 qiita"で検索をかけてみると、ほとんどの記事が式変形をベースに議論を展開している。さらに、いくつかの記事では変換行列を$\exp{(-2\pi \i nm)}$のように記述し、波数(周波数)を意識しないストーリーに仕上げていた。確かにCSに十分精通している場合、確固たるCSの基礎の上に理屈を組み上げることができるかもしれないが、そうでない場合"雲のように不安定な土台"の上に積み上げることになる(実体験)。この場合数日経つと忘れることはもちろん、理解の応用も難しいのではないか。多くの人によって愛用されている手法の場合は間違いなく何かしらのQ\&Aに関連しているはずで、それを実用したいならその背景情報は不可欠ではないだろうか。物理の考え方はこの課題を直接解決できる。さっきの$\exp{(-2\pi \i nm)}$を例にとると、なぜ虚数が使われるのか、なぜ$e$の肩はマイナスなのか、なぜ$\sin$や$\cos$だと不十分なのか、などなど数式にまつわる疑問に対し、感覚的に回答を与えられる。というか、数式を直感的に、自然現象を通して解釈することが物理の特徴なんです。この随筆では座標の基礎を出発点、Fourier変換を用いたカードの問題の解法を終点と定め、この二点を直感的な(なるべく数学に頼らない)理解で繋ぐことを目標とする。この目的のため、以降の式変形において厳密でない箇所が多々あるが、数学的理解は別の記事を参考にしてほしい。


\end{document}